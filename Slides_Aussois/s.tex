\documentclass[8pt,red, compress]{beamer}
\usepackage [french]{babel}
\usepackage [utf8]{inputenc}
\usepackage{fontenc}

\usepackage{amsmath,amsthm,amsfonts,amssymb,amsbsy,stmaryrd}
\usepackage{subeqnarray}

\usepackage{cases}

\usepackage{pifont}

\usepackage{fancyhdr}

\usepackage{lastpage}
\usepackage{subfigure}

\usepackage{wrapfig}

\usepackage{graphicx}
\usepackage{subfloat}
\usepackage{rotating}
\usepackage{tikz}

\usetikzlibrary{arrows}
\usetikzlibrary{calc}
\usetikzlibrary{positioning, shapes}


%\usepackage{minted}
\usepackage{color}

%\usepackage{tikz-uml}
% \usepackage{algorithm}
% \usepackage{algorithmic}


\newcommand<>{\fullsizegraphic}[1]{
  \begin{textblock*}{0cm}(-1cm,-3.78cm)
  \includegraphics[width=\paperwidth]{#1}
  \end{textblock*}
}


%% Symbole de fraction
\newcommand{\Frac}[2]{{\displaystyle \frac{\displaystyle #1}{\displaystyle #2}}}
\newcommand{\Prac}[2]{\displaystyle \genfrac{(}{)}{}{}{\displaystyle #1}{\displaystyle #2}}
\newcommand{\Crac}[2]{\displaystyle \genfrac{[}{]}{}{}{\displaystyle #1}{\displaystyle #2}}

\newcommand{\norme}[1]{\|#1\|}


\newcommand{\HRule}{\rule{\linewidth}{1mm}}

% Fonction math�matiques

\newcommand{\transposee}[1]{{\vphantom{#1}}^{\text{\tiny{\textsf T}}}{#1}}
%\newcommand{\argmin}{\mathop{\mathrm{argmin}}}
\newcommand{\argminn}{\mathop{\mathrm{argmin}}}
\newcommand{\lexicomin}{\mathop{\mathrm{lexicomin}}}
%\newcommand{\arg}{\mathop{\mathrm{arg}}}



\DeclareMathOperator{\rot}{rot}
\DeclareMathOperator{\sh}{sh}
\DeclareMathOperator{\ch}{ch}
%\DeclareMathOperator{\th}{th}
\DeclareMathOperator{\arcsh}{arcsh}
\DeclareMathOperator{\argth}{argth}
\DeclareMathOperator{\sign}{sign}




%%The Principal Value Integral symbol
\def\Xint#1{\mathchoice
   {\XXint\displaystyle\textstyle{#1}}%
   {\XXint\textstyle\scriptstyle{#1}}%
   {\XXint\scriptstyle\scriptscriptstyle{#1}}%
   {\XXint\scriptscriptstyle\scriptscriptstyle{#1}}%
   \!\int}
\def\XXint#1#2#3{{\setbox0=\hbox{$#1{#2#3}{\int}$}
     \vcenter{\hbox{$#2#3$}}\kern-.5\wd0}}
\def\ddashint{\Xint=}
\def\dashint{\Xint-}



% macro pour les symbols d'ensemble
%\nbOne
\def\nbOne{{\mathchoice{\rm 1\mskip-4mu l}{\rm 1\mskip-4mu l} {\rm 1 \mskip-4.5mu l}{\rm 1\mskip-5mu l}}}
%
%%  Les ensembles de nombres  C. Fiorio (fiorio�at�math.tu-berlin.de) 
%
\def\nbR{\ensuremath{\mathrm{I\!R}}} % IR
\def\nbN{\ensuremath{\mathrm{I\!N}}} % IN
\def\nbF{\ensuremath{\mathrm{I\!F}}} % IF
\def\nbH{\ensuremath{\mathrm{I\!H}}} % IH
\def\nbK{\ensuremath{\mathrm{I\!K}}} % IK
\def\nbL{\ensuremath{\mathrm{I\!L}}} % IL
\def\nbM{\ensuremath{\mathrm{I\!M}}} % IM
\def\nbP{\ensuremath{\mathrm{I\!P}}} % IP
%
% \nbOne : 1I : symbol one
\def\nbOne{{\mathchoice {\rm 1\mskip-4mu l} {\rm 1\mskip-4mu l}
{\rm 1\mskip-4.5mu l} {\rm 1\mskip-5mu l}}}
%
% \nbC   :  Nombres Complexes
\def\nbC{{\mathchoice {\setbox0=\hbox{$\displaystyle\rm C$}%
\hbox{\hbox to0pt{\kern0.4\wd0\vrule height0.9\ht0\hss}\box0}}
{\setbox0=\hbox{$\textstyle\rm C$}\hbox{\hbox
to0pt{\kern0.4\wd0\vrule height0.9\ht0\hss}\box0}}
{\setbox0=\hbox{$\scriptstyle\rm C$}\hbox{\hbox
to0pt{\kern0.4\wd0\vrule height0.9\ht0\hss}\box0}}
{\setbox0=\hbox{$\scriptscriptstyle\rm C$}\hbox{\hbox
to0pt{\kern0.4\wd0\vrule height0.9\ht0\hss}\box0}}}}
%
% \nbQ   : Nombres Rationnels Q
\def\nbQ{{\mathchoice {\setbox0=\hbox{$\displaystyle\rm
Q$}\hbox{\raise
0.15\ht0\hbox to0pt{\kern0.4\wd0\vrule height0.8\ht0\hss}\box0}}
{\setbox0=\hbox{$\textstyle\rm Q$}\hbox{\raise
0.15\ht0\hbox to0pt{\kern0.4\wd0\vrule height0.8\ht0\hss}\box0}}
{\setbox0=\hbox{$\scriptstyle\rm Q$}\hbox{\raise
0.15\ht0\hbox to0pt{\kern0.4\wd0\vrule height0.7\ht0\hss}\box0}}
{\setbox0=\hbox{$\scriptscriptstyle\rm Q$}\hbox{\raise
0.15\ht0\hbox to0pt{\kern0.4\wd0\vrule height0.7\ht0\hss}\box0}}}}
%
% \nbT   : T
\def\nbT{{\mathchoice {\setbox0=\hbox{$\displaystyle\rm
T$}\hbox{\hbox to0pt{\kern0.3\wd0\vrule height0.9\ht0\hss}\box0}}
{\setbox0=\hbox{$\textstyle\rm T$}\hbox{\hbox
to0pt{\kern0.3\wd0\vrule height0.9\ht0\hss}\box0}}
{\setbox0=\hbox{$\scriptstyle\rm T$}\hbox{\hbox
to0pt{\kern0.3\wd0\vrule height0.9\ht0\hss}\box0}}
{\setbox0=\hbox{$\scriptscriptstyle\rm T$}\hbox{\hbox
to0pt{\kern0.3\wd0\vrule height0.9\ht0\hss}\box0}}}}
%
% \nbS   : S
\def\nbS{{\mathchoice
{\setbox0=\hbox{$\displaystyle     \rm S$}\hbox{\raise0.5\ht0%
\hbox to0pt{\kern0.35\wd0\vrule height0.45\ht0\hss}\hbox
to0pt{\kern0.55\wd0\vrule height0.5\ht0\hss}\box0}}
{\setbox0=\hbox{$\textstyle        \rm S$}\hbox{\raise0.5\ht0%
\hbox to0pt{\kern0.35\wd0\vrule height0.45\ht0\hss}\hbox
to0pt{\kern0.55\wd0\vrule height0.5\ht0\hss}\box0}}
{\setbox0=\hbox{$\scriptstyle      \rm S$}\hbox{\raise0.5\ht0%
\hboxto0pt{\kern0.35\wd0\vrule height0.45\ht0\hss}\raise0.05\ht0%
\hbox to0pt{\kern0.5\wd0\vrule height0.45\ht0\hss}\box0}}
{\setbox0=\hbox{$\scriptscriptstyle\rm S$}\hbox{\raise0.5\ht0%
\hboxto0pt{\kern0.4\wd0\vrule height0.45\ht0\hss}\raise0.05\ht0%
\hbox to0pt{\kern0.55\wd0\vrule height0.45\ht0\hss}\box0}}}}
%
% \nbZ   : Entiers Relatifs Z
\def\nbZ{{\mathchoice {\hbox{$\sf\textstyle Z\kern-0.4em Z$}}
{\hbox{$\sf\textstyle Z\kern-0.4em Z$}}
{\hbox{$\sf\scriptstyle Z\kern-0.3em Z$}}
{\hbox{$\sf\scriptscriptstyle Z\kern-0.2em Z$}}}}
%%%% fin macro %%%%



\newcommand{\putidx}[1]{\index{#1}\textit{#1}}
% macro pour r�f�rencer les �quations

\newcommand{\refeq}[1]{(\ref{#1})}
\newcommand{\reffig}[1]{({\it cf} figure : \ref{#1})}
\newcommand{\refann}[1]{({\it cf} Annexe : \ref{#1})}


%\definecolor{darkgray}{gray}{.25}
\definecolor{gray}{gray}{.5}
\definecolor{lightgray}{gray}{.75}
%\definecolor{gradbegin}{rgb}{0,1,1}
%\definecolor{gradend}{rgb}{0,.1,.95}
%\newcommand{\newtexte}[1]{\textcolor{darkgray} {#1}}
\newcommand{\newtexte}[1]{{#1}}% macro pour les varibales favorites
% normal tangent
\def\n{{\hbox{\tiny{N}}}}
\def\t{{\hbox{\tiny{T}}}}
\def\ss{{\hbox{\tiny{S}}}}
\def\nt{\hbox{\tiny{NT}}}
\def\nsf{\hbox{\tiny{\textsf N}}}
\def\tsf{\hbox{\tiny{\textsf T}}}
\def\sigman{\sigma_{\n}}
\def\sigmat{\sigma_{\t}}
\def\sigmant{\sigma_{\nt}}
\def\epsn{\epsilon_{\n}}
\def\epst{\epsilon_{\t}}
\def\epsnt{\epsilon_{\nt}}
\def\eps{\epsilon}
\def\veps{\varepsilon}
\def\sig{\sigma}
\def\Rn{R_{\n}}
\def\Rt{R_{\t}}
\def\cn{c_{\n}}
\def\Cn{C_{\n}}
\def\ct{c_{\t}}
\def\Ct{C_{\t}}
\def\un{u_{\n}}
\def\ut{\buu_{\t}}
\def\uut{u_{\t}}
\def\unc{u_{\n}^c}
\def\utc{\buu_{\t}^c}
\def\vn{v_{\n}}
\def\vt{v_{\t}}
\def\rr{\hbox{\tiny{\textsf R}}}
\def\irr{\hbox{\tiny{\textsf{IR}}}}
\def\rn{r_{\n}}
\def\rt{\brr_{\t}}
\def\rnc{r_{\n}^c}
\def\rtc{\brr_{\t}^c}
\def\trn{\Tilde{r}_{\n}}
\def\trt{\Tilde{\brr}_{\t}}
\def\tr{\Tilde{\brr}}
\def\tv{\Tilde{\bvv}}
\def\vn{v_{\n}}
\def\vt{\bvv_{\t}}
\def\adh{\mathsf{adh}}
\def\adj{\hbox{\tiny{\textsf{adj}}}}
\def\adjc{\hbox{\tiny{\textsf{adjC}}}}
\def\adja{\hbox{\tiny{\textsf{adjA}}}}
\def\cc{\hbox{\tiny{\textsf C}}}
\def\ca{\hbox{\tiny{\textsf A}}}

%%    Unit�e
\def\mm{\,\mathsf{mm}}
\def\cm{\,\mathsf{cm}}
\def\m{\,\mathsf{m}}
\def\ms{\,\mathsf{m.s^{-1}}}
\def\mms{\,\mathsf{mm.s^{-1}}}
\def\Mpa{\,\mathsf{MPa}}
\def\Gpa{\,\mathsf{GPa}}
\def\Kg{\,\mathsf{Kg}}
\def\Hz{\,\mathsf{Hz}}
\def\kHz{\,\mathsf{kHz}}
\def\N{\,\mathsf{N}}
\def\kN{\,\mathsf{kN}}
\def\Nmmm{\,\mathsf{N.m^{-3}}}
\def\ds{d_{\hbox{\tiny{S}}}}
% domaines et frontieres
\def\om{\Omega}
\def\oma{\Omega^{\alpha}}
\def\omu{\Omega^1\cup \Omega^2}
\def\gc{\Gamma_c}
\def\omt{\omu \cup \gc}
% derivee partielle et gradient et divergence
\def\p{\partial}
\def\grad{\nabla}
\def\div{\mathop{\rm div}\nolimits}
%

%\DeclareTextSymbol{\deg}{T1}{6}
%\def\degre{\mathdegree}
%\newcommand{\degre}{\mathdegree}

\def\etc{\textit{etc}\ldots}
\newcommand{\mdegre}{\hbox{\text{\degre}}}

%\def\nscd{\textsf{\bfseries NSCD}}
%\def\nscd{\textsf{NSCD}}
\newcommand{\nscd}{\textsf{NSCD}}
%\Pisymbol{psy}{212} ou encore \Pisymbol{psy}{228}




%----------------------------------------------------------------------
%             Des chiffres avec des ronds autour
%----------------------------------------------------------------------
\def\nombrecercle#1{\def\taille{0.3}
                \put(0,0){#1}
                \put(0.08,0.08){\circle{\taille}}}



\def\ae#1{\stackrel{\mbox{\scriptsize a.e.}}{#1}}
%\def\argmin{\mathop{\rm argmin}}


\def\eqref#1{{\rm (\ref{#1})\/}}
\def\indicfon{\mathord{\rm i}}       %indicator function
\def\p{\mathord{\rm proj}}
\def\N{\mathord{\rm N}}
% \def\prosca#1#2{#1\cdot#2}
\def\prosca#1#2{\langle #1,#2\rangle}
\def\qedtext{\mbox{}\hfill$\Box$}
\def\qedmath{\eqno\Box}

\def\s{{$\mathcal{S}$}}
\def\somme{\mathop{\textstyle\sum}}
\def\somme{\mathop{\textstyle\sum}}
\def\submoins{_{\scriptscriptstyle-}}
\def\subplus{_{\scriptscriptstyle+}}
\def\T{\mathord{\rm T}}

%----------------------------------------------------------------------
%             Macro M Jean 
%----------------------------------------------------------------------

\def\Real{\mbox{I\hspace{-.15em}R}}
\def\Integer{\mbox{I\hspace{-.15em}N}}
\def\Bunit{\mbox{I\hspace{-.15em}B}}
\def\real{\mbox{\scriptsize I\hspace{-.15em}R}}
\def\bunit{\mbox{\scriptsize I\hspace{-.15em}B}}
\def\IL{\mbox{\scriptsize I\hspace{-.15em}L}}
\def\Indic{\mbox{\large $\psi$}}
\def\bfxi{\mbox{$\xi$ \hspace{-1.1em} $\xi$}}
%\def\bfXi{\mbox{$\Xi$ \hspace{-1.1em} $\Xi$}}
\def\RunR{\mathcal R}
\def\RunRN{\mathcal R_{N}}
\def\RunRT{\mathcal R_{T}}
\def\RunS{\mathcal S}
\def\RunSN{\mathcal S_{N}}
\def\RunST{\mathcal S_{T}}
\def\RunU{\mathcal U}
\def\RunUN{\mathcal U_{N}}
\def\RunUT{\mathcal U_{T}}
\def\RunUP{\mathcal U'}
\def\RunUPN{\mathcal U'_{N}}
\def\RunUPT{\mathcal U'_{T}}
\def\RunJ{\mathcal J}
\def\RunW{\mathcal W}
\def\RunF{f}
\def\RunFa{f_{1}}
\def\RunFb{f_{2}}
\def\RunFP{f'}
\def\RunV{v}
\def\RunVP{v'}
\def\EspF{\mathcal F}
\def\EspV{\mathcal V}
%%%%
\catcode`\�=13
\def�{\'e}
\catcode`\�=13
\def�{\`e}
\catcode`\�=13
\def�{\`a}
\catcode`\�=13
\def�{\c c}
\def\N{\mbox{I\hspace{ -.15em}N}}
\def\Z{\mbox{Z\hspace{ -.3em}Z}}
\def\Q{\mbox{l\hspace{ -.47em}Q}}
\def\R{\mbox{l\hspace{ -.15em}R}}
\def\F{\mbox{l\hspace{ -.15em}F}}
\def\E{\mbox{l\hspace{ -.15em}E}}
\def\LMGC90{{\small \it LMGC90 }}
\def\NSCD{{\small \it NSCD }}
\def\CHIC{{\small \it CHIC }}
\def\half{{\frac{_{1}}{^{2}}}}
\def\12T{{\frac{_{1}}{^{2T}}}}

\def\geq{\geqslant}
\def\leq{\leqslant}
\def\ge{\geqslant}
\def\le{\leqslant}


\begingroup
\count0=\time \divide\count0by60 % Hour
\count2=\count0 \multiply\count2by-60 \advance\count2by\time
% Min
\def\2#1{\ifnum#1<10 0\fi\the#1}
\xdef\isodayandtime{\the\year-\2\month-\2\day\space\2{\count0}:%
\2{\count2}}
\endgroup

%---------------------------------------------------------------------
%             Redaction note environnement B. Brogliato
%----------------------------------------------------------------------
\makeatletter

{\newtheorem{ndr1bb}{\textbf{\textsc{Redaction note B.B.}}}[section]}

\newenvironment{ndrbb}%
{%
\noindent\begin{ndr1bb}\hrule\vspace{1em}%
\ttfamily\small
}%
{%
\begin{flushright}%
%\vspace{-1.5em}\ding{111}
\end{flushright}%
\vspace{-1.5em}\hrule
\end{ndr1bb}%
}
%----------------------------------------------------------------------
%             Redaction note environnement V.ACARY
%----------------------------------------------------------------------
% Faut etre fou pour s'amuser a pondre des notes pareilles

{\newtheorem{ndr1va}{\textbf{\textsc{\footnotesize Redaction note V.A.}}}[section]}

\newenvironment{ndrva}%
{%
\noindent\begin{ndr1va}\hrule\vspace{1em}%
\ttfamily\small \  \\
\noindent}%
{%
$ $ \\
\hrule
\end{ndr1va}%
}
\makeatother








% ----------------DEFINITIONS-----------------
% 

 \def\II{\mathop{{\rm I}\mskip-3.0mu{\rm I}}\nolimits}




% -----------------------------------
 \def\c{\mathop{{\rm 1}\mskip-10.0mu{\rm C}}\nolimits}
 \def\C{\mathop{{\rm 1}\mskip-10.0mu{\rm C}}\nolimits}
 \def\ZZ{\mathaccent23Z}
% 

\newcommand{\ie}{{\textit{i.e.}}}


%\def\sgn{\mbox{\rm sgn}}
\DeclareMathOperator{\sgn}{sgn}
\DeclareMathOperator{\proj}{proj}
\DeclareMathOperator{\prox}{prox}
\DeclareMathOperator{\argmin}{argmin}

\newcommand{\RR}{\mbox{\rm $I\!\!R$}}
\newcommand{\NN}{\mbox{\rm $I\!\!N$}}


% ---------------- MMC -----------------
% 

\newcommand{\contract}{{\,:\,}}

\newcommand{\scontract}{{\,{\Bar\otimes}\,}}
\newcommand{\tcontract}{{\,{\Bar{\Bar{\Bar\otimes}}}\,}}


\newcommand{\DP}[2]{\displaystyle \frac{\partial {#1}}{\partial {#2}}}

% %\newtheorem{definition}{Definition}
% \newtheorem{proposition}{Proposition}
% \newtheorem{lemma}{Lemma}

% \newtheorem{claim}{Claim}
% \newtheorem{remark}{Remark}
% \newtheorem{assumption}{Assumption}
% \newtheorem{example}{Example}
% \newtheorem{conjecture}{Conjecture}
% \newtheorem{corollary}{Corollary}
% \newtheorem{OP}{OP}
% \newtheorem{problem}{Problem}
% \newtheorem{theorem}{Theorem}


\def\dt{{\rm d}t}
\def\dv{{\rm d}v}
\def\di{{\rm d}i}
\def\dI{{\rm d}I}
\def\dU{{\rm d}U}


\def\nat{{\hbox{\tiny{nat}}}}
\def\nor{{\hbox{\tiny{nor}}}}
\def\fb{\hbox{\tiny{\textsf FB}}}
\def\vione{{\hbox{\tiny{vi-1}}}}
\def\vitwo{{\hbox{\tiny{vi-2}}}}
\def\qvitwo{{\hbox{\tiny{qvi-2}}}}
\def\acone{{\hbox{\tiny{ac-1}}}}
\def\actwo{{\hbox{\tiny{ac-2}}}}


\usepackage{wasysym}



\setbeamertemplate{theorem}[ams style]
\setbeamertemplate{theorems}[numbered]
\graphicspath{{./figure-Hots/},{./figure-BiolSyst/}}


%\input{../macro-mjean.tex}
% %hideothersubsections
% \usetheme[hideothersubsections,width=2.5cm]{Goettingen}
% %\useoutertheme[headline=empty]{miniframes}

\usetheme[]{Montpellier}
\makeatletter
\useoutertheme[height=0pt, width=0cm]{sidebar}
{\usebeamercolor{structure}}
\setbeamertemplate{sidebar canvas \beamer@sidebarside}[vertical shading][top=structure.fg!25,bottom=structure.fg!10]

%%% Local Variables: 
%%% mode: latex
%%% TeX-master: t
%%% End: 

\setbeamercolor{block title}{fg=red!90}
\setbeamercolor{frame title}{fg=red!90}
\setbeamerfont{itemize/enumerate body}{family=\sffamily, size={\fontsize{8}{8}}}
\setbeamerfont{itemize/enumerate subbody}{family=\sffamily, size={\fontsize{8}{8}}}

\title[Pour une épi-revue en Mécanique \\$ $ \\ {Vincent Acary, Mathias Legrand}]{{\huge Pour une épi-revue en Mécanique}}
\author{Vincent Acary, Mathias Legrand \\ [2mm]
Aussois 2018 – Colloque National MECAMAT}
\date{\today}

\setbeamertemplate 
{footline} 
{\quad\hfill\strut\insertsection\quad--\quad\insertframenumber/\inserttotalframenumber\strut\quad\quad} 


\includeonly{%
  introduction
}



%\newtheorem{defn}{Definition}
%\renewcommand{\thedefn}{\arabic{defn}}
% \newtheorem{thm}[defn]{Theorem}
% \newtheorem{corr}[defn]{Corollary}
% \newtheorem{ass}[defn]{Assumption}
% \newtheorem{lem}[defn]{Lemma}
% \newtheorem{rem}[defn]{Remark}
% \newtheorem{hypo}[defn]{Hypotheses}
% \newtheorem{exmp}[defn]{Example}
% \newtheorem{prop}[defn]{Proposition}
\newcommand{\diag}{\mbox{\rm diag}}
\newcommand{\co}{\overline{\mathit{co}}}
\newcommand{\rect}{\overline{\mathit{rect}}}
\newcommand{\newb}{g}

\renewcommand{\tr}[1]{\textcolor{red}{#1}}
\usepackage{smartdiagram}


\def\toto{date}
\def\tutu{item}
\makeatletter
\NewDocumentCommand{\smartdiagramx}{r[] m}{%
    \StrCut{#1}{:}\diagramtype\option
    \IfStrEq{\diagramtype}{priority descriptive diagram}{% true-priority descriptive diagram
        \pgfmathparse{subtract(\sm@core@priorityarrowwidth,\sm@core@priorityarrowheadextend)}
        \pgfmathsetmacro\sm@core@priorityticksize{\pgfmathresult/2}
        \pgfmathsetmacro\arrowtickxshift{(\sm@core@priorityarrowwidth-\sm@core@priorityticksize)/2}
        \begin{tikzpicture}[every node/.style={align=center,let hypenation}]
          \foreach \smitem [count=\xi] in {#2}{\global\let\maxsmitem\xi}
        
          \foreach \smitem [count=\xi] in {#2}{%
            \StrCut{\smitem}{:}\toto\tutu
            \edef\col{\@nameuse{color@\xi}}
            \node[description,drop shadow](module\xi)
            at (0,0+\xi*\sm@core@descriptiveitemsysep) {\tutu};
            \draw[line width=\sm@core@prioritytick,{\@nameuse{color@3}}]
            ([xshift=-\arrowtickxshift pt]module\xi.base west)--
            ($([xshift=-\arrowtickxshift pt]module\xi.base west)-(\sm@core@priorityticksize pt,0)$) node[left]{\toto};
          }%
        \coordinate (A) at (module1);
        \coordinate (B) at (module\maxsmitem);
        \CalcHeight(A,B){heightmodules}
        \pgfmathadd{\heightmodules}{\sm@core@priorityarrowheightadvance}
        \pgfmathsetmacro{\distancemodules}{\pgfmathresult}
        \pgfmathsetmacro\arrowxshift{\sm@core@priorityarrowwidth/2}
        \begin{pgfonlayer}{background}
        \node[priority arrow,rotate=180,transform shape] at ([xshift=-\arrowxshift pt]module\maxsmitem.north west){};
        \end{pgfonlayer}
        \end{tikzpicture}
    }{}% end-priority descriptive diagram
}%

\makeatother


\begin{document}
{
%\usebackgroundtemplate{\includegraphics[width=\paperwidth]{Hartmann23.pdf}}
 \begin{frame}[plain,noframenumbering]
    \titlepage
  \end{frame}
}


\begin{frame}
  \frametitle{Qu'est ce qu'une épi-revue?}
  \begin{block}{Source d'information }
    CCSD : Centre pour la communication scientifique directe ((UMS CNRS 3668 INRIA, Université de Lyon.)
    \href{https://www.ccsd.cnrs.fr/2013/01/episciences-de-quoi-sagit-il/}{https://www.ccsd.cnrs.fr/2013/01/episciences-de-quoi-sagit-il/}
  \end{block}

  
  Le préfixe « épi » signifie « sur », « au-dessus ». (épi-centre, épigraphe)

  \begin{block}{Une définition}
    Une épi-revue est une  revue électronique en libre accès, alimentées par les articles déposés dans les archives ouvertes telles que HAL ou ArXiv, et non publiés par ailleurs.
  \end{block}




\end{frame}
\begin{frame} \frametitle{Qu'est ce qu'une épi-revue?}
  \begin{block}{En pratique}
  \begin{itemize}
  \item  Les comités éditoriaux des épi-revues organisent l’activité d’évaluation et de discussion scientifique des prépublications soumises ou sélectionnées.
  \item Les épi-revues peuvent ainsi être considérées comme une « sur-couche » aux archives ouvertes ; elles y apportent une valeur ajoutée en apposant la caution scientifique d’un comité éditorial à chaque article retenu.
  \item Les épi-revues peuvent être soit des nouveaux titres, soit des titres existants dont la politique d’accès permet de se joindre à la plate-forme.
  \item Aucune cession de droit n’est signée avec les auteurs qui conservent leurs droits patrimoniaux sur leurs articles.
  \end{itemize}
\end{block}

\begin{block}{Enjeu}
  \begin{itemize}
  \item Permettre de réaliser des revues à moindre coût et de
    mettre en œuvre le libre accès aux versions électroniques des
    articles.
  \item Proposer ainsi une alternative aux modèles économiques existants, sans pour autant se placer en concurrence avec les éditeurs
  \end{itemize}
\end{block}
\end{frame}

\begin{frame} \frametitle{La plate-forme Episciences.org}
  Une initiative du CCSD fortement soutenu par le CNRS et l'INRIA basée sur les archives ouvertes HAL, arXiv, CWI.
  \begin{block}{Objectifs}
    \begin{itemize}
    \item Proposer un outil complet pour la gestion de la revue, son hébergement et la diffusion de ses contenus.
    \item Proposer un ensemble de fonctionnalités personnalisables :
      \begin{itemize}
      \item Hébergement et personnalisation du site web (design,
        contenu,\ldots)
      \item Paramétrage de la revue (grille de relecture,
        volumes, rubriques \ldots)
      \item Gestion des utilisateurs et de leurs
        rôles (secrétaire de rédaction, webmaster, relecteur,
        relecteur invité, \ldots)
      \item Gestion et suivi des relectures
      \item Envoi de
        mail (relance,\ldots)
      \item Publication des articles
      \end{itemize}
    \end{itemize}
  \end{block}

  \begin{block}{Aujourd'hui}
    \begin{itemize}
    \item Hébergement de 10 épi-revues
    \item Discussion pour une systèmes d'archive ouverte européenne
    \end{itemize}
  \end{block}

\end{frame}

\begin{frame} \frametitle{Et en Mécanique ?}
  \begin{block}{État des lieux}
    \begin{itemize}
    \item Très peu de revues en accés ouverts sans frais de publications (ou frais modérés)
      \begin{itemize}
      \item Journal of Mechanics of Materials and Structures
      \item Memocs : Mathematics and Mechanics of Complex Systems
      \item Latin American Journal of Solids and Structures
      \item Journal of the Brazilian Society of Mechanical Sciences and Engineering
      \item Archives of Mechanics
      \item Quelques revues nationales (Hongrie, Jordanie, \ldots.)
      \end{itemize}
    \item La plupart des revues sont editées pas des editeurs commerciaux.
      \begin{itemize}
      \item Même les sociétés savantes ont laissé le travail d'édition aux éditeurs commerciaux
        (IUTAM, EUROMECH, Académie des sciences, \ldots.)
      \end{itemize}
    \item Le plupart des revues ont un mode de fonctionnement engageant peu la communauté internationale.
      Prédominance des revue anglo-saxonnes, comités éditoriaux fermés, \ldots.
    \end{itemize}
  \end{block}
\end{frame}
  
\begin{frame}
  \frametitle{Pourquoi pas une épi-revue en Mécanique?}
  \begin{block}{Motivations}
    \begin{itemize}
    \item Besoin d'un modèle alternatif.
    \item Peu de revues ouvertes
    \item Soutien technique du CCSD
    \end{itemize}
  \end{block}
  \begin{block}{Discussion et engagement de la communauté}
    Au niveau national, démarrage d'une discussion sur Loomio\\[2mm]
    \href{https://www.loomio.org/g/yPoIWNc5/for-an-epi-journal-in-mechanics}{https://www.loomio.org/g/yPoIWNc5/for-an-epi-journal-in-mechanics}
  \end{block}
  
\end{frame}

\begin{frame}
  \frametitle{Pourquoi pas une épi-revue en Mécanique?}
  \begin{block}{Élaboration d'un manifeste}
    \begin{itemize}
    \item Need for an epijournal in Mechanical engineering
    \item Scope of the epijournal
    \item Review procedure
    \item Editorial management
    \item Sustainable science
    \item Ethics
    \end{itemize}
  \end{block}
  \begin{block}{Périmètre d'une épi-revue}
    La question du périmètre doit être débattue:
    \begin{itemize}
    \item Revue à large spectre.
      \begin{itemize}
      \item par exemple A/ Mécanique du solide et B/ Mécanique des fluides
      \item création ensuite de sous sections.
      \end{itemize}
    \item Revues à périmètre resserré porté pas une communauté bien identifiée.
    \end{itemize}
  \end{block}
\end{frame}


\begin{frame}
  \frametitle{Pourquoi pas une épi-revue en Mécanique?}
  \begin{block}{Management éditorial}
    \begin{itemize}
    \item A ton besoin d'un éditeur en chef ?
    \item Comité scientifique international de premier plan
    \item Comité d'éditeurs associés avec un rôle effectif.
      \begin{itemize}
      \item nommés ou élus (par la communauté ou par le comité
        scientifique)
      \item mandat de durée déterminée.
      \end{itemize}
    \end{itemize}
  \end{block}
  \begin{block}{Procédure de relecture}
    \begin{itemize}
    \item Processus de relecture anonyme ou non, doublement anonyme
    \item Discussion des articles sur le site du journal des sa soumission
    \end{itemize}
  \end{block}
\end{frame}
\begin{frame}
  \frametitle{Pourquoi pas une épi-revue en Mécanique?}
  \begin{block}{Aspects éthiques}
    \begin{itemize}
    \item Rappel des régles européennes en terme d'éthique et de déontologie.
    \item Veiller à éviter le plagiat (outil de scan)
    \item Ne pas imposer des citations des articles du journal
    \item Création d'un comité d'éthique
    \item Sanction et communication en cas de non respect de ses régles. 
    \end{itemize}
  \end{block}
  \begin{block}{Science durable et reproductible}
    \begin{itemize}
    \item Conservation et accés aux données d'expérience
      \begin{itemize}
      \item expérience de laboratoire \\
        ``A manifesto for reproducible science'' Nature Human Behaviour 1, 2017
      \item expérience numérique \\
        ``Sustainable computational science: the ReScience initiative'' \href{https://arxiv.org/abs/1707.04393}{https://arxiv.org/abs/1707.04393}
      \end{itemize}
    \end{itemize}
    
  \end{block}
\end{frame}

\begin{frame}
  \frametitle{Difficultés}
  \begin{itemize}
  \item Prise de conscience large de la communauté
  \item Réunir et motiver un comité scientifique déjà bien occupé
  \item Appui des sociétés savantes.
  \item Indexation et reconnaissance du journal. (Recrutement des jeunes checheurs.)
  \end{itemize}
\end{frame}
% {
\begin{frame}[plain,noframenumbering]
  \centerline{\Huge \textcolor{black}{Merci de votre attention}}

\vfill
  Planches disponibles \href{https://github.com/vacary/epijournalmechanics/tree/master/Slides_Aussois}{https://github.com/vacary/epijournalmechanics/tree/master/Slides\_Aussois}
\end{frame}
% }

%\include{PWL}


\end{document}

%%% Local Variables: 
%%% mode: latex
%%% TeX-master: t
%%% End: 
